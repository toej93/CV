%%%%%%%%%%%%%%%%%%%%%%%%%%%%%%%%%%%%%%%%%
% Medium Length Professional CV
% LaTeX Template
% Version 2.0 (8/5/13)
%
% This template has been downloaded from:
% http://www.LaTeXTemplates.com
%
% Original author:
% Trey Hunner (http://www.treyhunner.com/)
%
% Important note:
% This template requires the resume.cls file to be in the same directory as the
% .tex file. The resume.cls file provides the resume style used for structuring the
% document.
%
%%%%%%%%%%%%%%%%%%%%%%%%%%%%%%%%%%%%%%%%%

%----------------------------------------------------------------------------------------
%	PACKAGES AND OTHER DOCUMENT CONFIGURATIONS
%----------------------------------------------------------------------------------------
\documentclass{resume} % Use the custom resume.cls style
\usepackage{hyperref}
\usepackage{etaremune}
\usepackage{tabularx}
%\pagestyle{plain}
%\pagestyle{myheadings}
\usepackage{fancyhdr}
\pagestyle{fancy}
\fancyhf{}
\lhead{\textit{Curriculum Vitae -- Jorge Torres --} \thepage} 
\renewcommand{\headrulewidth}{0pt}


\newenvironment{list1}{
  \begin{list}{\ding{113}}{%
      \setlength{\itemsep}{0in}
      \setlength{\parsep}{0in} \setlength{\parskip}{0in}
      \setlength{\topsep}{0in} \setlength{\partopsep}{0in} 
      \setlength{\leftmargin}{0.17in}}}{\end{list}}

\usepackage[left=0.75in,top=0.65in,right=0.75in,bottom=0.6in]{geometry} % Document margins
\name{Jorge Torres} % Your name


%\address{B-1 \\ II , U.P. 208016} % Your address
%\address{123 Pleasant Lane \\ City, State 12345} % Your secondary addess (optional)
%\address{(+1)~7hh~9~8486 \\ x@ii.yzac.in} % Your phone number and email

\begin{document}

%----------------------------------------------------------------------------------------
%	CONTACT
%----------------------------------------------------------------------------------------

\vspace{-1cm}
%\begin{rSection}{CONTACT}
\rule{\textwidth}{0.1cm} \\ \\
\begin{tabular}{@{}p{2in}p{4in}}
191 W. Woodruff Ave             & {\it Phone:}  (614) 822-7264\\            
Physics Research Building   & {\it Email:}  torresespinosa.1@osu.edu \\         
The Ohio State University  & {\it Website:} \url{u.osu.edu/torresespinosa.1}  \\       
Columbus, OH  43210 USA  \\     
\end{tabular}
%\end{rSection}
%\vspace{0.25cm}

%----------------------------------------------------------------------------------------
%	RESEARCH INTERESTS
%---------------------------------------------------------------------------------------

\begin{rSection}{RESEARCH PROFILE}
Experimental particle astrophysics PhD candidate at The Ohio State University working with the Askaryan Radio Array (ARA). Interested in ultra-high energy (UHE) neutrino astronomy, specifically the simulation, and data analysis of radio-based Antarctic neutrino telescopes, as well as the development of new UHE neutrino detection techniques.
\end{rSection}
%\vspace{0.25cm}

%----------------------------------------------------------------------------------------
%	EDUCATION
%----------------------------------------------------------------------------------------

\begin{rSection}{EDUCATION}
\textbf{The Ohio State University}, Columbus, Ohio USA \hfill Fall 2015--Spring 2021 (Expected)\\
\vspace*{-.15in}
\begin{list1}
\item[] Ph.D. in Physics--Advisor: Prof. Amy Connolly
\item [] Master of Science in Physics, July 2017
\end{list1}

\textbf{Universidad de Colima}, Colima, Mexico. \hfill 2011--2015\\
\vspace*{-.15in}
\begin{list1}
\item[] Bachelor of Science in Physics--Advisor: Alfredo Aranda
\end{list1}
\end{rSection}
%\vspace{0.25cm}

%----------------------------------------------------------------------------------------
%	AWARDS
%---------------------------------------------------------------------------------------
\begin{rSection}{AWARDS}
 \begin{itemize}
 \item Ohio SuperComputer Center Statewide Users Group Conference Talk Award \hfill 10/2017
 \item APS Division of Astrophysics Travel Grant to attend the APS April Meeting \hfill 04/2019

 \end{itemize}
\end{rSection}
%----------------------------------------------------------------------------------------
%	RESEARCH EXPERIENCE
%----------------------------------------------------------------------------------------

\begin{rSection}{RESEARCH EXPERIENCE}
{\bf The Ohio State University}, Columbus, OH USA \hfill {\bf August 2015 -- present} \\
 {\em Ph.D. Student}, Ultra-High Energy Neutrino Astrophysics
\begin{itemize}
\vspace*{.05in}
\item Developer in the simulation and analysis efforts in Askaryan Radio Array (ARA) collaboration to detect ultra-high energy neutrinos.
\item Helped in the construction and realization of the experiment T-576 to detect radio-frequency waves bouncing off particle showers. The experiment was carried out at SLAC National Accelerator Laboratory.
\item Member of the InIceMC simulation group, aimed at improving simulations of radio-based UHE in-ice neutrino experiments. 

\end{itemize}

\end{rSection}


%----------------------------------------------------------------------------------------
%	PUBLICATIONS
%----------------------------------------------------------------------------------------
\begin{rSection}{PUBLICATIONS}
\begin{etaremune}%[leftmargin=0.64cm]
 \item ``Constraints on the Diffuse Flux of Ultra-High Energy Neutrinos from Four Years of Askaryan Radio Array Data in Two Stations" \\
 P. Allison {\it et. al.} ({\bf co-author})\\   To be submitted to Physical Review D (2020).  \href{https://arxiv.org/abs/1912.00987}{[arXiv:1912.00987]} 
 
 \item ``Observation of Radar Echoes From High-Energy Particle Cascades" \\
 S. Prohira {\it et. al.} (incl. \textbf{J. A. Torres})\\    Accepted to Physical Review Letters (2020).  \href{https://arxiv.org/abs/1910.12830}{[arXiv:1910.12830]} 
  \item ``Long-baseline horizontal radio-frequency transmission through polar ice" \\
 P. Allison {\it et. al.} for the ARA Collaboration (incl. \textbf{J. A. Torres})\\    Submitted to Journal of Glaciology (2019). \href{https://arxiv.org/abs/1908.10689}{[arXiv:1908.10689]}
  \item ``NuRadioMC: Simulating the radio emission of neutrinos from interaction to detector" \\
 C. Glaser {\it et. al.} (incl. \textbf{J. A. Torres})\\   Eur.Phys.J. C80 (2020) no.2, 77. \href{https://arxiv.org/abs/1906.01670}{[arXiv:1906.01670]} 
  \item ``Suggestion of Coherent Radio Reflections from an Electron-Beam Induced Particle Cascade" \\
 S.Prohira {\it et. al.} (incl. \textbf{J. A. Torres})\\    Accepted to to PRD (2019). \href{https://arxiv.org/abs/1810.09914}{[arXiv:1810.09914]} 
 
 \end{etaremune}
\end{rSection}

%----------------------------------------------------------------------------------------
%	SCIENTIFIC TALKS
%----------------------------------------------------------------------------------------

\begin{rSection}{SCIENTIFIC TALKS}
\begin{etaremune}[]
\item Talk, Graduate Student Summer Seminar Series, Columbus OH. \hfill 2019/07/17 \\
{\em Ultra-High Energy Neutrinos: Physics and Detection} 
\item Talk, Radio-Workshop, DESY (Zeuthen), Germany. \hfill 2019/06/19 \\
{\em Validation of in-ice simulations} 
\item Talk, APS April Meeting, Denver CO. \hfill 2019/04/15 \\
{\em Simulations of radio-based Ultra-High Energy (UHE) in-ice neutrino experiments} 
\item Talk, Ohio Supercomputer Center Statewide Users Group Conference, Columbus, OH. \hfill 2018/04/05 \\
{\em The role of HPC in the radiodetection of astrophysical neutrinos} 
%\item Talk, Second Colima Winter School on High Energy Physics and Workshop, Colima, Mexico. \hfill 2018/01/12 \\
%{\em Ultra-High Energy Neutrinos: Physics and Radio-detection} 
\item Talk, Computing in High Energy Astropart. Phys. Research 2016, Columbus OH. \hfill 2016/05/26 \\
{\em The BuckArray: detecting cosmic rays with cellphones} 
 \end{etaremune}
\end{rSection}
\vspace{-0.10cm}

%----------------------------------------------------------------------------------------
%	RELEVANT SKILLS
%----------------------------------------------------------------------------------------
\begin{rSection}{RELEVANT SKILLS}
\begin{tabular}{@{}l l l@{}}
 Programming/Software & & C++, C, Python, BASH,  \LaTeX, Git\\ 
\end{tabular}
\end{rSection}

%----------------------------------------------------------------------------------------
%	PUBLICATIONS
%----------------------------------------------------------------------------------------
\begin{rSection}{TEACHING}
Teaching Assistant, ``Physics 1201:E\&M, Optics and Quantum Mechanics", OSU \hfill {Spring 2018--Summer 2018}\\
Teaching Assistant, ``Physics 1250: Mech, Thermo, Waves", OSU \hfill {Fall 2015--Spring 2017}\\
\end{rSection}
\vspace{-0.35cm}

%----------------------------------------------------------------------------------------
%	OUTREACH
%----------------------------------------------------------------------------------------
\begin{rSection}{OUTREACH AND SERVICE}
Delegate, Council of Graduate Students (CGS), OSU \hfill August 2019--present\\
Talk (high school students), Instituto Heisenberg, Colima, Mexico  \hfill May 2019\\
Volunteer Poster Judge, Ohio Supercomputer Center  \hfill April 2018--present\\
Counsel member for the Society for Women in Physics (SWiP), OSU \hfill August 2017--December 2018\\
Coordinator for \href{u.osu.edu/aspire}{ASPIRE} Workshop for High School Girls, OSU \hfill July 2017--present\\
\end{rSection}
\vspace{-0.30cm}
 %----------------------------------------------------------------------------------------
%	MENTORSHIP
%----------------------------------------------------------------------------------------

\begin{rSection}{MENTORSHIP}
\begin{table}[h]
\begin{tabularx}{\textwidth}{l X}
{\bf Undergraduate Students:}  & Ian Best, Hannah Hassan \\
\end{tabularx}
\end{table}
\end{rSection}

\end{document}